\documentclass{article}

\usepackage[margin=1in]{geometry}

\usepackage{lmodern}
\usepackage{amsmath}
\usepackage{amssymb}
\usepackage{upgreek}
\usepackage{pgfplots}
\usepackage{qtree}
\usepackage{calc}
\usepackage[normalem]{ulem}
\usepackage[utf8]{inputenc}
\usepackage[T1]{fontenc}

\begin{document}

\section{Teoria dos Números}
Propriedades dos números inteiros $\mathbb{Z}$ com respeito às operações elementares.
\begin{tabbing}
  Equação diofantina: \= Equação polinomial que permite a duas ou mais variáveis \\
  \> assumirem apenas valores inteiros.
\end{tabbing}



\section{Conceitos Fundamentais}

\subsection{Divisão Euclidiana}
Sejam $a, b \in \mathbb{Z}$ com $b \neq 0$, a divisão euclidiana de $a$ por $b$ consiste na identidade
\[ a = b \cdot q + r \qquad q, r \in \mathbb{Z} \> \land \> 0 \leq r < b \]


\subsection{Divisibilidade}
Sejam $a, b \in \mathbb{Z}$ com $b \neq 0$, dizemos que $b$ divide $a$, denotando $b \mid a$, se
\[ \exists \> c \in \mathbb{Z}: \enspace a = b \cdot c \]
Propriedades:
\begin{itemize}
  \item $\forall \> a \in \mathbb{Z}: \enspace a \mid 0$
  \item $\forall \> a \in \mathbb{Z}: \enspace \pm 1 \mid a$
  \item $\forall \> a \in \mathbb{Z}: \enspace \pm a \mid a$
  \item $\forall \> c \in \mathbb{Z}: \enspace a \mid b \implies ac \mid bc$
  \item $\forall \> x, y \in \mathbb{Z}: \enspace a \mid b \> \land \> a \mid c \implies a \mid (bx + cy)$
  \item $\forall \> a, b \in \mathbb{Z}: \enspace a \mid b \> \land \> b \mid a \implies b = \pm a$
\end{itemize}


\subsection{Máximo Divisor Comum}
Sejam $a, b \in \mathbb{Z}$ com $(a, b) \neq (0, 0)$, o máximo divisor comum de $a$ e $b$ é um inteiro $d$ tal que
\begin{gather*}
  d \mid a \> \land \> d \mid b \\[5pt]
  \forall \> d': \enspace d' \mid a \> \land \> d' \mid b \implies d' \mid d
\end{gather*}
\vspace{-10pt}
\begin{tabbing}
  \uline{Lema}: \= Sejam $a, b \in \mathbb{Z}$ com $(a,b) \neq (0,0)$, e $q, r \in \mathbb{Z}$ com $a = b \cdot q + r$. \\
  \> O $\text{mdc}(a,b)$, se existe, é igual a $\text{mdc}(b, r)$.
\end{tabbing}
\vspace{7pt}
\uline{Identidade de Bézout}: Sejam $a, b \in \mathbb{Z}$ com $(a,b) \neq (0,0)$, então
\vspace{-5pt}
\[ \exists \, \alpha, \beta \in \mathbb{Z}: \enspace \alpha \cdot a + \beta \cdot b = \text{mdc}(a,b) \]
\uline{Lema de Euclides}: Sejam $a,b,c \in \mathbb{Z}$ com $a,b,c \neq 0$. Se $a|bc$ e $\text{mdc}(a,b) = 1$, então $a|c$. \\[10pt]
Propriedades: Sejam $a,b,c \in \mathbb{Z}$ com $a,b,c \neq 0$
\begin{itemize}
  \item $\text{mdc}(a,c) = \text{mdc}(b,c) \iff \text{mdc}(a b,c) = 1$
  \item $\text{mdc}(a,b) = d \iff \text{mdc} \left( \dfrac{a}{d}, \dfrac{b}{d} \right) = 1$
  \item $a \mid c \:\land\: b \mid c \implies \left( \dfrac{ab}{\text{mdc}(a,b)} \right) \bigg | \, c$
  \item $(a \mid c \:\land\: b \mid c \:\land\: \text{mdc}(a,b) = 1) \implies ab \mid c$
\end{itemize}


\subsection{Mínimo Multiplo Comum}
Sejam $a, b \in \mathbb{Z}$ com $(a, b) \neq (0, 0)$, o mínimo multiplo comum de $a$ e $b$ é um inteiro $m$ tal que
\begin{gather*}
  a \mid m \:\land\: b \mid m \\[5pt]
  \forall \> m': \enspace a \mid m' \:\land\: b \mid m' \implies m \mid m'
\end{gather*}
\uline{Teorema}: $\forall \, a, b \in \mathbb{Z}, (a, b) \neq (0, 0): \enspace \text{mmc}(a,b) = \dfrac{ab}{\text{mdc}(a,b)}$


\subsection{Números Primos}
Um número $p$ é primo se os únicos divisores de $p$ são $\pm1$ e $\pm p$.
\begin{tabbing}
  \uline{Lema}: \= Seja $p \in \mathbb{Z}$ primo, e $x_1, \hdots, x_n \in \mathbb{Z}$. \\
  \> Se $p \mid (x_1 \cdot \hdots \cdot x_n)$, então $p \mid x_i$ para ao menos algum $i \in [1, n] \subset \mathbb{Z}$.
\end{tabbing}
\uline{Teorema}: Qualquer número natural $n \geq 2$ é produto de um conjunto \uline{único} e finito de números primos.
\begin{tabbing}
  \uline{Corolário}: \= Seja $n \in \mathbb{Z}$ com $n \neq 0, \pm 1$. \\
  \> Sejam $p_1, \hdots, p_n \in \mathbb{Z}$ primos. \\
  \> Sejam $h_1, \hdots, h_n \in \mathbb{Z}$ maiores que $0$. \\
  \> $n$ pode ser escrito como $n = \pm \left( p_1^{h_1} + \hdots + p_n^{h_n} \right)$
\end{tabbing}



\end{document}
