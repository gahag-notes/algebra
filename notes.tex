\documentclass{article}

\usepackage[margin=1in]{geometry}

\usepackage{amsmath}
\usepackage{amssymb}
\usepackage{calc}
\usepackage{lmodern}
\usepackage{pgfplots}
\usepackage{qtree}
\usepackage{slashed}
\usepackage{upgreek}
\usepackage{xfrac}
\usepackage[normalem]{ulem}
\usepackage[utf8]{inputenc}
\usepackage[T1]{fontenc}

\begin{document}

\section{Conceitos Fundamentais}

\subsection{Divisão Euclidiana}
Sejam $a, b \in \mathbb{Z}$ com $b \neq 0$, a divisão euclidiana de $a$ por $b$ consiste na identidade
\[ a = b \cdot q + r \qquad q, r \in \mathbb{Z} \> \land \> 0 \leq r < b \]


\subsection{Divisibilidade}
Sejam $a, b \in \mathbb{Z}$ com $b \neq 0$, dizemos que $b$ divide $a$, denotando $b \mid a$, se
\[ \exists \> c \in \mathbb{Z}: \enspace a = b \cdot c \]
Propriedades:
\begin{itemize}
  \item $\forall \> a \in \mathbb{Z}: \enspace a \mid 0$
  \item $\forall \> a \in \mathbb{Z}: \enspace \pm 1 \mid a$
  \item $\forall \> a \in \mathbb{Z}: \enspace \pm a \mid a$
  \item $\forall \> c \in \mathbb{Z}: \enspace a \mid b \implies ac \mid bc$
  \item $\forall \> x, y \in \mathbb{Z}: \enspace a \mid b \> \land \> a \mid c \implies a \mid (bx + cy)$
  \item $\forall \> a, b \in \mathbb{Z}: \enspace a \mid b \> \land \> b \mid a \implies b = \pm a$
\end{itemize}


\subsection{Máximo Divisor Comum}
Sejam $a, b \in \mathbb{Z}$ com $(a, b) \neq (0, 0)$, o máximo divisor comum de $a$ e $b$ é um inteiro $d$ tal que
\begin{gather*}
  d \mid a \> \land \> d \mid b \\[5pt]
  \forall \> d': \enspace d' \mid a \> \land \> d' \mid b \implies d' \mid d
\end{gather*}
\vspace{-10pt}
\begin{tabbing}
  \uline{Lema}: \= Sejam $a, b \in \mathbb{Z}$ com $(a,b) \neq (0,0)$, e $q, r \in \mathbb{Z}$ com $a = b \cdot q + r$. \\
  \> O $\text{mdc}(a,b)$, se existe, é igual a $\text{mdc}(b, r)$.
\end{tabbing}
\vspace{7pt}
\uline{Identidade de Bézout}: Sejam $a, b \in \mathbb{Z}$ com $(a,b) \neq (0,0)$, então
\vspace{-5pt}
\[ \exists \, \alpha, \beta \in \mathbb{Z}: \enspace \alpha \cdot a + \beta \cdot b = \text{mdc}(a,b) \]
\uline{Lema de Euclides}: Sejam $a,b,c \in \mathbb{Z}$ com $a,b,c \neq 0$. Se $a|bc$ e $\text{mdc}(a,b) = 1$, então $a|c$. \\[10pt]
Propriedades: Sejam $a,b,c \in \mathbb{Z}$ com $a,b,c \neq 0$
\begin{itemize}
  \item $\text{mdc}(a,c) = \text{mdc}(b,c) \iff \text{mdc}(a b,c) = 1$
  \item $\text{mdc}(a,b) = d \iff \text{mdc} \left( \dfrac{a}{d}, \dfrac{b}{d} \right) = 1$
  \item $a \mid c \:\land\: b \mid c \implies \left( \dfrac{ab}{\text{mdc}(a,b)} \right) \bigg | \, c$
  \item $(a \mid c \:\land\: b \mid c \:\land\: \text{mdc}(a,b) = 1) \implies ab \mid c$
\end{itemize}


\subsection{Mínimo Multiplo Comum}
Sejam $a, b \in \mathbb{Z}$ com $(a, b) \neq (0, 0)$, o mínimo multiplo comum de $a$ e $b$ é um inteiro $m$ tal que
\begin{gather*}
  a \mid m \:\land\: b \mid m \\[5pt]
  \forall \> m': \enspace a \mid m' \:\land\: b \mid m' \implies m \mid m'
\end{gather*}
\uline{Teorema}: $\forall \, a, b \in \mathbb{Z}, (a, b) \neq (0, 0): \enspace \text{mmc}(a,b) = \dfrac{ab}{\text{mdc}(a,b)}$


\subsection{Fatoração}
\uline{Lema}: Seja $n = ab \in \mathbb{Z}$ com $n \neq 0, \pm 1$, então $a \leq \lfloor \sqrt{n} \rfloor \>\lor\> b \leq \lfloor \sqrt{n} \rfloor$.


\subsection{Números Primos}
Um número $p$ é primo se os únicos divisores de $p$ são $\pm1$ e $\pm p$.
\begin{tabbing}
  \uline{Lema}: \= Seja $p \in \mathbb{Z}$ primo, e $x_1, \hdots, x_n \in \mathbb{Z}$. \\
  \> Se $p \mid (x_1 \cdot \hdots \cdot x_n)$, então $p \mid x_i$ para ao menos algum $i \in [1, n] \subset \mathbb{Z}$.
\end{tabbing}
\vspace{10pt}
\uline{Teorema}: Qualquer número natural $n \geq 2$ é produto de um conjunto \uline{único} e finito de números primos.
\begin{tabbing}
  \uline{Corolário}: \= Seja $a \in \mathbb{Z}$ com $a \neq 0, \pm 1$. \\
  \> Sejam $p_1, \hdots, p_n \in \mathbb{Z}$ primos. \\
  \> Sejam $h_1, \hdots, h_n \in \mathbb{Z}$ maiores que $0$. \\
  \> $a$ pode ser escrito como $a = \pm \left( p_1^{h_1} \cdot \hdots \cdot p_n^{h_n} \right)$
\end{tabbing}
\begin{tabbing}
  \uline{Corolário}: \= Seja $a, b \in \mathbb{Z}$ com $a$ e $b \neq 0, \pm 1$. \\
  \> Sejam $\forall\: i : h_i, k_i \geq 0$, e $p_1, \hdots, p_n \in \mathbb{Z}$ primos tais que \\[3pt]
  \> $a = \pm \enspace p_1^{h_1} \cdot \hdots \cdot p_n^{h_n}$ \\[3pt]
  \> $b = \pm \enspace p_1^{k_1} \cdot \hdots \cdot p_n^{k_n}$ \\[3pt]
  \> Então: \\
  \>\begin{minipage}{\linewidth}
    \begin{itemize}
      \item $\text{mdc}(a,b) = p_1^{d_1} \cdot \hdots \cdot p_n^{d_n}$, onde $d_i = \text{min}(h_i, k_i)$
      \item $\text{mmc}(a,b) = p_1^{d_1} \cdot \hdots \cdot p_n^{d_n}$, onde $d_i = \text{max}(h_i, k_i)$
    \end{itemize}
  \end{minipage}
\end{tabbing}
\vspace{10pt}
\uline{Teorema}: Há um número infinito de números primos. \\[5pt]
\uline{Corolário}: Seja $p \in \mathbb{Z}$ primo com $p > 0$, então $\sqrt{p} \in \mathbb{Q}$.\\[10pt]
\uline{Teorema}: Não há forma polinomial que gere \uline{apenas} números primos. \\[10pt]
\uline{Teorema}: Sejam $a, b \in \mathbb{N}^+$ com $\text{mdc}(a,b) = 1$, então a sequência ${\left( an + b \right)}_{n=0}^{\infty}$ contém infinitos primos. \\[10pt]
A função para o número de primos menores que $x \in \mathbb{R}$ é
\[ \pi(x) \sim \frac{x}{\ln x} \]

\pagebreak

\subsubsection{Números de Fermat}
Os números de Fermat são dados pela função
\begin{gather*}
  F: \mathbb{N}^+ \to \mathbb{N}^+ \\
  F(n) = 2^{2^n} + 1
\end{gather*}
\uline{Teorema}: Nem todos números de Fermat são primos. \\[5pt]
\uline{Teorema}: Seja $a \geq 2 \in \mathbb{Z}$ e $a^2 + 1$ primo. Então $a$ é par e $n = 2^m$.
\begin{tabbing}
  \uline{Teorema}: \= $\forall \: k \in \mathbb{Z}, n \in \mathbb{N}^+: \text{mdc}(F(n), F(n + k)) = 1$. \\
  \> Ou seja, \uline{todos} números de Fermat são co-primos entre si.
\end{tabbing}
\uline{Corolário}: Como $F(1), \hdots, F(n)$ são co-primos, entre seus fatores há ao menos $n$ números primos distintos.

\subsubsection{Números de Mersenne}
Os números de Mersenne são dados pela função
\begin{gather*}
  M: \mathbb{P} \to \mathbb{N}^+ \\
  M(p) = 2^p - 1
\end{gather*}
\uline{Teorema}: Nem todos números de Mersenne são primos. \\[5pt]
\uline{Teorema}: Seja $a \in \mathbb{Z}$ com $a \geq 1$. Então $a^n - 1$ é primo se e somente se $a = 2$ e $n$ é primo.



\section{Congruências}

\subsection{Relações de Equivalência}
Uma relação sobre um conjunto $A$ é um subconjunto $R \subset A \times A$. \\
Dizemos que $a R b$ se $(a,b) \in R$. \\[5pt]
Uma relação \uline{pode} ter as seguintes propriedades:
\begin{itemize}
  \item Reflexividade: se $\forall \: a \in A: aRa$.
  \item Simetria: se $\forall \: a,b \in A: aRb \implies bRa$.
  \item Transitividade: $\forall \: a,b,c \in A: aRb \land bRc \implies aRc$.
  \item Antissimetria: se $\forall \: a,b \in A: aRb \land bRa \implies a = b$.
  \item Totalidade: se $\forall \: a,b \in A: aRb \oplus bRa$.
\end{itemize}
\uline{Definição}: Uma relação $R$ sobre $A$ é de \uline{equivalência} se ela é reflexiva, simétrica e transitiva.


\pagebreak


\subsection{Classes de Equivalência}
Seja $a \in A$ e $R$ uma relação de equivalência sobre $A$. Definimos a classe de equivalência de $a$ como
\[ {[a]}_R := \{ x \in A \mid aRx \} = \{ x \in A \mid xRa \} \]
Propriedades:
\begin{itemize}
  \item $\forall \: a \in A: a \in {[a]}_R$
  \item ${[a]}_R = {[b]}_R \iff aRb$
  \item ${[a]}_R \cap {[b]}_R = \varnothing \iff a\slashed{R}b$
  \item As classes de equivalência de um conjunto formam uma partição deste: $\forall \: A: A = \bigsqcup\limits_{a \in A} {[a]}_R$
\end{itemize}
Seja $R$ uma relação de equivalência sobre $A$. Denotamos o conjunto das classes de equivalência de $R$
\[ A_{/R} := \{ {[a]}_R \mid a \in A \} \]


\subsection{Congruência}
Seja $m \in \mathbb{Z}$ com $m > 1$. Dizemos que $a$ é congruente $b$ módulo $m$ se $m \mid (a - b)$. Denota-se
\[ a \equiv_m b \]
\uline{Teorema}: Para qualquer $m > 1$, $\equiv_m$ forma uma relação de equivalência sobre $\mathbb{Z}$.
\begin{itemize}
  \item $\forall \: a \in \mathbb{Z}: a \equiv_m a$.
  \item $\forall \: a,b \in \mathbb{Z}: a \equiv_m b \implies b \equiv_m a$.
  \item $\forall \: a,b,c \in \mathbb{Z}: a \equiv_m b \land b \equiv_m c \implies a \equiv_m c$.
\end{itemize}
\vspace{5pt}
Propriedades:
\begin{itemize}
  \item $a \equiv_m 0 \iff m \mid a$.
  \item $a \equiv_m b \iff -a \equiv_m -b$.
  \item $a \equiv_m b \:\land\: a' \equiv_m b' \implies (a + a') \equiv_m (b + b')$.
  \item $a \equiv_m b \:\land\: a' \equiv_m b' \implies (a \cdot a') \equiv_m (b \cdot b')$.
  \item $\forall \: k \neq 0 \in \mathbb{Z}: a \equiv_m b \iff ka \equiv_m kb$.
\end{itemize}
\vspace{-2pt}
\begin{tabbing}
  \uline{Teorema}: \= Seja $m \in \mathbb{Z}$ com $m > 1$. Então $\mathbb{Z}_{/m} = \{ {[0]}_m, {[1]}_m, \hdots {[m - 1]}_m \}$ \\[5pt]
  \> Portanto, $\big| \mathbb{Z}_{/m} \big| = m$.
\end{tabbing}
\uline{Corolário}: Seja $p(x)$ um polinômio com coeficientes inteiros. Então $a \equiv_m b \implies p(a) \equiv_m p(b)$.


\end{document}
