\documentclass{article}

\usepackage[margin=1in]{geometry}

\usepackage{lmodern}
\usepackage{amsmath}
\usepackage{amssymb}
\usepackage{upgreek}
\usepackage{pgfplots}
\usepackage{qtree}
\usepackage{calc}
\usepackage[normalem]{ulem}
\usepackage[utf8]{inputenc}
\usepackage[T1]{fontenc}

\begin{document}

\section{Teoria dos Números}
Propriedades dos números inteiros $\mathbb{Z}$ com respeito às operações elementares.
\begin{tabbing}
  Equação diofantina: \= Equação polinomial que permite a duas ou mais variáveis \\
  \> assumirem apenas valores inteiros.
\end{tabbing}



\section{Algoritmos Fundamentais}

\subsection{Divisão Euclidiana}
Sejam $a, b \in \mathbb{Z}$ com $b \neq 0$, a divisão euclidiana de $a$ por $b$ consiste na identidade
\[ a = b \cdot q + r \qquad q, r \in \mathbb{Z} \> \land \> 0 \leq r < b \]

\subsection{Divisibilidade}
Sejam $a, b \in \mathbb{Z}$ com $b \neq 0$, dizemos que $b$ divide $a$, denotando $b\,|\,a$, se
\[ \exists \> c \in \mathbb{Z}: \enspace a = b \cdot c \]
Propriedades:
\begin{itemize}
  \item $\forall \> a \in \mathbb{Z}: \enspace a\,|\,0$
  \item $\forall \> a \in \mathbb{Z}: \enspace \pm 1\,|\,0$
  \item $\forall \> a \in \mathbb{Z}: \enspace \pm a\,|\,a$
  \item $\forall \> c \in \mathbb{Z}: \enspace a\,|\,b \implies ac\,|\,bc$
  \item $\forall \> x, y \in \mathbb{Z}: \enspace a\,|\,b \> \land \> a\,|\,c \implies a\,|\,(bx + cy)$
  \item $\forall \> a, b \in \mathbb{Z}: \enspace a\,|\,b \> \land \> b\,|\,a \implies b = \pm a$
\end{itemize}

\subsection{Máximo Divisor Comum}
Sejam $a, b \in \mathbb{Z}$ com $(a, b) \neq (0, 0)$, o máximo divisor comum de $a$ e $b$ é um inteiro $d$ tal que
\begin{gather*}
  d \,|\, a \> \land \> d \,|\, b \\[5pt]
  \forall \> d': \enspace d' \,|\, a \> \land \> d' \,|\, b \implies d' \,|\, d
\end{gather*}
\vspace{-10pt}
\begin{tabbing}
  \uline{Lema}: \= Sejam $a, b \in \mathbb{Z}$ com $(a,b) \neq (0,0)$, e $q, r \in \mathbb{Z}$ com $a = b \cdot q + r$. \\[5pt]
  \> O $\text{mdc}(a,b)$, se existe, é igual a $\text{mdc}(b, r)$.
\end{tabbing}


\end{document}
